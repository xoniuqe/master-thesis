%------------------ Präambel ---------------------------------------------------
\documentclass[envcountsame, envcountchap, deutsch]{i-studis}

\usepackage[utf8]{inputenc}

\usepackage[a4paper]{geometry}
\usepackage[english, ngerman]{babel}

\usepackage[pdftex]{graphicx}
\usepackage{epstopdf}

\usepackage{listings}

\usepackage[german, ruled, vlined]{algorithm2e}
\usepackage{amssymb, amsfonts, amstext, amsmath}
\usepackage{array}
\usepackage[skip=10pt]{caption}
\usepackage[usenames, dvipsnames]{color}
\usepackage[pdftex, plainpages=false]{hyperref}
\usepackage{textcomp}
\usepackage{hyperref}
\usepackage{csquotes}

\usepackage{filecontents}

%\usepackage[numbers]{natbib} 
%\usepackage[english]{babel} 
%\usepackage[     %backend=biber,      natbib=true,     style=numeric,     sorting=none ]{biblatex}

\begin{filecontents*}{eprint-hal.dbx}
	\ProvidesFile{eprint-hal.dbx}[2018/09/26 HAL/TEL eprints]
	\DeclareDatamodelFields[type=field,datatype=verbatim]{arxiv,hal}
	\DeclareDatamodelEntryfields{hal}
	\DeclareDatamodelFields[type=field,datatype=literal]{arxivclass}
	\DeclareDatamodelEntryfields{arxivclass}
\end{filecontents*}
	
\usepackage[
    backend=biber,
    style=numeric,%authoryear-icomp,
    sortlocale=de_DE,
    natbib=true,
    url=true, 
    doi=true,
	datamodel=eprint-hal,
    eprint=true	
]{biblatex}

\usepackage{hyperref}

\DeclareFieldFormat{hal}{%
  \mkbibacro{HAL}\addcolon\space
  \ifhyperref
    {\href{https://hal.archives-ouvertes.fr/#1}{\nolinkurl{#1}}}
    {\nolinkurl{#1}}}

\DeclareFieldAlias{eprint:hal}{hal}
\DeclareFieldAlias{eprint:HAL}{eprint:hal}

\renewbibmacro*{eprint}{%
  \printfield{hal}%
  \newunit\newblock
  \iffieldundef{eprinttype}
    {\printfield{eprint}}
    {\printfield[eprint:\strfield{eprinttype}]{eprint}}}


\addbibresource{literatur.bib}

\usepackage{makeidx}
\usepackage{multicol}
\makeindex

\pagestyle{myheadings}
\setlength{\textheight}{1.1\textheight}

\lstset{
	basicstyle=\scriptsize\ttfamily,
	commentstyle=\scriptsize\ttfamily\color{Gray},
	identifierstyle=\scriptsize\ttfamily,
	keywordstyle=\scriptsize\ttfamily,
	stringstyle=\scriptsize\ttfamily,
	tabsize=4,
	numbers=left,
	numberstyle=\tiny,
	numberblanklines=false,
	frame=single,
	framesep=3mm,
	framexleftmargin=7mm,
	xleftmargin=10mm,
	linewidth=144mm,
	captionpos=b,
}

%-- styles for code

\definecolor{codegreen}{rgb}{0,0.6,0}
\definecolor{codegray}{rgb}{0.5,0.5,0.5}
\definecolor{codepurple}{rgb}{0.58,0,0.82}
\definecolor{backcolour}{rgb}{0.95,0.95,0.92}

\lstdefinestyle{cpp}{
	backgroundcolor=\color{backcolour},   
    commentstyle=\color{codegreen},
    keywordstyle=\color{magenta},
    numberstyle=\tiny\color{codegray},
    stringstyle=\color{codepurple},
    basicstyle=\ttfamily\footnotesize,
    breakatwhitespace=false,         
    breaklines=true,                 
    keepspaces=true,                 
    numbers=left,       
    numbersep=5pt,                  
    showspaces=false,                
    showstringspaces=false,
    showtabs=false,                  
    tabsize=2,
}

%------------------ Manuelle Silbentrennung ------------------------------------
\hyphenation{Ele-men-tar-ob-jek-te ab-ge-tas-tet Aus-wer-tung House-holder-Matrix Least-Squares-Al-go-ri-th-men}


%------------------ Titelseite -------------------------------------------------
\begin{document}

\title{Implementierung eines Verfahrens zu effizienten Berechnung oszillierender Integrale}
\subtitle{Implementation of a method for the efficient calculation of oscillatory integrals}

\author{Tobias Arens}

\supervisor{Titel Vorname Nachname}

\address{Mehren}
\submitdate{08.10.2022}

%------------------ Projektart -------------------------------------------------
\project{Master-Abschlussarbeit}

\mytitlepage
\nocite{*}

%------------------ Vorwort, Kurzfassung, Verzeichnisse ------------------------
\frontmatter
\kurzfassung

%In der Kurzfassung soll in kurzer und prägnanter Weise der wesentliche Inhalt der Arbeit beschrieben werden. Dazu zählen vor allem eine kurze Aufgabenbeschreibung, der Lösungsansatz sowie die wesentlichen Ergebnisse der Arbeit. Ein häufiger Fehler für die Kurzfassung ist, dass lediglich die Aufgabenbeschreibung (d.h. das Problem) in Kurzform vorgelegt wird. Die Kurzfassung soll aber die gesamte Arbeit widerspiegeln. Deshalb sind vor allem die erzielten Ergebnisse darzustellen. Die Kurzfassung soll etwa eine halbe bis ganze DIN-A4-Seite umfassen.

%Hinweis: Schreiben Sie die Kurzfassung am Ende der Arbeit, denn eventuell ist Ihnen beim Schreiben erst vollends klar geworden, was das Wesentliche der Arbeit ist bzw. welche Schwerpunkte Sie bei der Arbeit gesetzt haben. Andernfalls laufen Sie Gefahr, dass die Kurzfassung nicht zum Rest der Arbeit passt.

In dieser Arbeit wird das in \citetitle{gasperini:hal-03209144} vorgestellte Verfahren implementiert. Basierend auf der Matlab-Implementierung der Autoren wurde eine hinsichtlich der Benutzbarkeit und Laufzeit eine verbesserte C++Implementierung umgesetzt.
Die Laufzeit der neuen Implementierung konnte im Vergleich nahezu halbiert werden und bei gleicher Genauigkeit der berechneten Integrale.  
Diese Verbesserungen wurden mithilfe von CPU-Parallelisierungen erreicht. 
Um die Implementierung einsetzen zu können werden ein Python-Modul sowie Matlab Mex-Funktionen bereitgestellt.


\kurzfassungEN

In this paper, the method described in \citetitle{gasperini:hal-03209144}.
is implemented. Based on the Matlab implementation of the
authors, an improved C++ implementation with respect to usability and runtime was implemented. The runtime of the new implementation has been
almost halved in comparison and with the same accuracy of the calculated
integrals. In order to be able to use this in a meaningful way, a Python module as well as a
Matlab mex-functions are provided.							% Kurzfassung/Abstract
\tableofcontents										% Inhaltsverzeichnis
\listoffigures											% Abbildungsverzeichnis (optional)
\listoftables											% Tabellenverzeichnis (optional)
\lstlistoflistings										% Listings (optional)


\lstset{inputpath=code}
%------------------ Kapitel ----------------------------------------------------
\mainmatter
\chapter{Einleitung und Problemstellung}

%Begonnen werden soll mit einer Einleitung zum Thema, also Hintergrund und Ziel erläutert werden.

%Weiterhin wird das vorliegende Problem diskutiert: Was ist zu lösen, warum ist es wichtig, dass man dieses Problem löst und welche Lösungsansätze gibt es bereits. Der Bezug auf vorhandene oder eben bisher fehlende Lösungen begründet auch die Intention und Bedeutung dieser Arbeit. Dies können allgemeine Gesichtspunkte sein: Man liefert einen Beitrag für ein generelles Problem oder man hat eine spezielle Systemumgebung oder ein spezielles Produkt (z.B. in einem Unternehmen), woraus sich dieses noch zu lösende Problem ergibt.
\section{Zielsetzung}

Basierend auf der Arbeit \citetitle{gasperini:hal-03209144}\cite{gasperini:hal-03209144} soll in dieser Arbeit das daraus resultierende Verfahren neu implementiert werden.
Die ursprüngliche Implementierung in Matlab funktioniert zwar, weißt aber Unzulänglichkeiten hinsichtlich der Benutzbarkeit und auch der Performanz auf.
So kann diese Lösung lediglich in Matlab verwendet werden und bietet keine Schnittstelle für Nutzer an. Des weiteren wurden keine Maßnahmen zur Optimierung der implementierten Lösung ergriffen.

In dieser Arbeit sollen diese beiden Aspekte gelöst werden. Zum einen wird eine Implementierung angestrebt welche Paralellisierungstechniken verwendet um das Laufzeitverhalten zu verbessern, sowie eine API anbietet welche einfach verwendet werden kann.
Darüber hinaus soll die Benutzbarkeit mithilfe eines Matlab-Plugins und eines Python-Moduls erleichtert werden.
Wichtig dabei ist, dass die Implementierung keinen Verlust hinsichtlich der Genauigkeit der berechneten Integrale erleidet.



\section{Aufbau der Arbeit}

In Kapitel \ref{algo} dieser Arbeit wird zunächst das in \cite{gasperini:hal-03209144} entwickelte Verfahren vorgestellt und einige wesentliche Teilaspekte erläutert.
Danach werden in Kapitel \ref{desgin} die Entwurfsbedingungen erläutert und auf die konkreten Anforderungen sowie die geplante Architektur der Umsetzung eingegangen.
In \ref{plan} wird die Planung hinsichtlich der Optimierung erläutert, welche in Kapitel \ref{analysis} ausgewertet und präsentiert werden.
Im Kapitel \ref{impl} werden einige Details der Implementierung hervorgehoben, sowie die verwendeten Technologien vorgestellt.
\chapter{Das Verfahren}

In diesem Kapitel wird das zu implemnetierende VErfahren vorgestellt.
\section{Accoustic single-layer integral operator}

Welches Problem löst das Verfahren:
\begin{equation*}
    \Delta
    I_{r,\Delta}(k) = \int_{\Delta}^{}  \frac{e^{ik(\left\lVert r-r'\right\rVert + \theta \cdot r')}}{\left\lVert r-r'\right\rVert} \,dr'
\end{equation*}

Wie wird es im Paper angegangen:

Und dann auf die relevanten Mathematischen "Grundlagen" eingehen die verwendet werden um das Verfahren implementieren:

\section{Mathematische Grundlagen}
\subsection{Cauchy-Integral-Theorem}



\subsection{Gauss-Laguerre Quadratur}

\subsection{Steepest Descent Method}




\chapter{Vorgehensweise}\label{plan}

In diesem Kapitel wird darauf eingegangen, wie die in Kapitel \ref{Performance} definierten Ziele erreicht werden.

Damit das Ziel der schnellen Performanz erreicht wird, wird die folgende Vorgehensweise genutzt.

Zu Beginn wird der Algorithmus ohne Rücksicht auf Perfomanzkriterien implementiert.
Dabei wird mithilfe von Unittests sichergestellt, dass sich die Implementierung korrekt verhält. 
Die korrekten Eregbnisse einzelner Tests können mithilfe der vorliegenden Matlab-Implementierung verifiziert werden.

Nachdem in diesem ersten Schritt ein korrekter Algorithmus vorliegt, kann mit der Optimierung begonnen werden.
Dazu werden folgende Methoden verwendet, auf welche in den jeweiligen Unterkapiteln eingegangen wird.

\begin{itemize}
    \item Profiling
    \item Hotpath-Analyse
    \item Benchmarking
    \item Parallelisierung isolierter Bereiche
\end{itemize}

\section{Profiling}

Mithilfe der Profiling-Werkzeuge der Entwicklungsumgebung \textit{Visual Studio} lässt sich das Laufzeitverhalten einer Anwendung analysieren.
In dieser Arbeit wird dabei das Messen der Geschwindigkeit als primäre Metrik genutzt. 


\section{Hotpath-Analyse}

Die Ergebnisse des Profilings ermöglichen es, den sogenannten Hotpath (häufig auch Critical-Path genannt) eines Algorithmus zu finden.
Das Ziel dieser Betrachtung ist es, die Funktion(en) zu finden, welche den größten Beitrag zu der gemessenen Laufzeit hat. 
Die so gefundenen Funktionen sind die besten Kandidaten für Optimierungsmaßnahmen, da jede andere Funkion weniger zu der Laufzeit beiträgt.


\section{Benchmarking}

%Mithilfe der Google Bibliothek \textit{Benchmark} werden 

Um die Laufzeit der Implementierung beurteilen zu können, wird auf zweierlei Arten von Benchmarks gesetzt.

Zum einen automatisierte Benchmarks, welche vorallem die API-Endpunkte mit verschiedenen Testdaten ausführen und das Laufzeitverhalten messen.
Diese Ergebnisse werden genutzt, um während der Entwicklung schnelle Rückmeldung über einzelne Maßnahmen zu erhalten.

Die zweite Variante der Benchmarks ist das manuelle Ausführen von API-Aufrufen, welche mithilfe von Timingfunktionen gemessen werden.
Diese werden für Tests genutzt, bei denen eine Automatisierung mit erheblichem Mehraufwand verbunden wäre.
Diese Vorgehensweise wird beispielsweise beim Vergleich der Matlabimplementierung mit den Matlab-Modulen angewandt. 

\section{Parallelisieren isolierter Bereiche}

Einige mögliche Parallelisierung lassen sich direkt aus der Problemstellung ablesen. 
So ist beispielsweise die Berechnung des zweidimensionalen Falls von zwei Dreiecken $\Delta_1$ und $\Delta_2$, abgesehen von 
der Wellenzahl $k$, des Beobachtungspunktes $\theta$ und der Anzahl der Gauss-Laguerre-Knoten, unabhängig voneinander.
Diese Berechnungen lassen sich dementsprechend relativ unproblematisch parallelisieren, da keine klassischen Threading-Konflikte auftreten können.
Da der Zugriff auf diese invarianten Parameter lediglich lesend ist, können die Zugriffe auf diese Ressourcen sogar ohne Synchronisierungen durchgeführt werden.

Ähnliches gilt für das Berechnen der einzelene Schichten im zweidimensionalen Fall. Die Ergebnisse einzelner Schichten beeinflussen sich gegenseitig nicht.

\section{Sonstige Maßnahmen}

Desweiteren werden mithilfe von Compiler- und Linker-Flags einige Optimierungen hinsichtlich der Laufzeit ausgewählt.
\chapter{Entwurf}
In diesem Kapitel wird auf verschiedene Aspekte der Entwurfsphase eingegangen.
Zum einen wird dargelegt mit welcher Herangehensweise der Algorithmus umgesetzt wird und zum anderen werden die Entscheidungen über die verwendeten Technologien begründet.

\section{Anforderungen}

Ziel dieser Arbeit ist es das in \cite*[]{gasperini:hal-03209144} beschriebene Verfahren zu implementieren.
Dabei sind drei Anforderungsbereiche auszumachen, welche in diesem Abschnitt behandelt werden.


\subsection{Korrektheit}

Die trivialste Anforderung ist den beschriebenen Algorithmus korrekt zu implementieren.
Diese Anforderung ist erfüllt, wenn die Rechenresultate der Implementierung eine Genauigkeit der gleichen Größenordnung wie die MatLab-Implementierung von \cite*[]{gasperini:hal-03209144}.
Dort werden in Kapitel 6.1 (Tabelle 3) für die Parameter 
\begin{equation}
    A = \begin{pmatrix}
        0 & 0 \\
        2 & 0 \\
        0 & 2 \\
    \end{pmatrix}, b = \begin{pmatrix}
        0 \\ -1 \\ 0
    \end{pmatrix},
    r = \begin{pmatrix}
        0.6 \\ 0 \\0 
    \end{pmatrix},
    \theta = \begin{pmatrix}
        1 \\ 0 \\
    \end{pmatrix}
\end{equation}

die relativen Fehler des implementierten Verfahrens mit der Matlab $integral$ Funktion für verschiedene Wellenzahlen $k$ verglichen.
\begin{table}[ht]
    \centering
    \begin{tabular}{|l|l|}
    \hline
    k & Relativer Fehler \\ 
    \hline \hline
    100  & $1.44 × 10^{-15}$ \\
    500  &  $1.15 × 10^{-14}$ \\
    1000 &  $9.71 × 10^{-5}$ \\
    3000 &  $5.11 × 10^{-7}$ \\
    5000 & $1.31 × 10^{-8}$ \\  \hline
    \end{tabular}    
    \caption{Genauigkeit der Implementierung von \cite*[]{gasperini:hal-03209144}}
\end{table} 
Das Kriterium der Korrektheit wird als erfüllt angesehen wenn die Implementierung höchstens Fehler der gleichen Größenordnung liefert.

\pagebreak

\subsection{Bedienbarkeit}

Die Anforderung der Bedienbarkeit wird im Rahmen dieser Arbeit wie folgt definiert:

\begin{enumerate}
    \item Es wird eine C++ Bibliothek bereitgstellt welche Mithilfe des Build-Systesm CMake eingebunden werden kann. 
    \item Es wird eine oder mehrere Matlab-Mex Funktionen ausgeliefert welche, den Algorithmus in Matlab benutzbar machen. 
    \item Es wird ein Python-Modul bereitgesellt mit welchem die 1D und 2D Algorithmen genutzt werden können.
\end{enumerate}

Dabei ist es nicht das Ziel die beschriebenen Pakete offiziel auszuliefern, das eingehändige Compilieren der Bilbiothek und ggf. der Module ist erforderlich.

\subsection{Geschwindigkeit}\label{Performance}


Der implementierte Algorithmus darf nicht langsamer als die bereitgestellte Matlab-Implementierung sein.
Zu diesesm Zweck werden einige Auswertungen mit randomisierten Daten mithilfe von MatLab ausgeführt und direkt mit der vorhandenen MatLab-Implementierung verglichen.



\section{Architektur}



\begin{itemize}
    \item Iteratives Design 
    \item In Tests hat sich gezeigt, dass zu viel Design zu schlechtere Performance führt (z.B. nutzen von Parameterobjekten oder zu viel indirektion für besser lesbaren code)
    \item Functor Objekte erläutern
    \item Numerische Integraiton mit GSL muss noch gelöst werden
\end{itemize}

%\section{Multithreading betrachtung des Algorithmus}

%Was kann man wo Parallelisieren?

%\section{Thread pool bzw. object pool}


\chapter{Implementierung}

In diesem Kapitel wird dargelegt wie der Algorithmus implementiert wird. Zu Beginn wird erläutert warum welche Technologie zum Einsatz kommt und welche Alternativen es gibt.
\section{Verwendete Technologien}

In diesem Abschnit werden die verwendeten Technologien vorgestellt. 


\subsection{GNU Scientific Library}

Die GNU Scientific Library (GSL) ist eine Sammlung von numerischen Funktionen\cite{gsl}. Die GSL wird unter der GNU General Public License veröffentlicht und ist in der Programmiersprache C geschrieben.

Die GSL bietet unter anderem Funktionen für \textit{BAsic linear Algebrar Subprograms}(BLAS), verschiedene Interpolationsalgorihtmen, Monte-Carlo-Algorithmen, Implementierung für Fast Forier Tranformationen und die für diese 
Arbeit benötigten Algorithmen für die numerische Integration.
GSL bietet eine Reimplementierung des QUADPACK\cite{quadpack}, eine Fortran Bibiliothek für numerische Integration.
Diese umfasst eine mehrere verschiedene Verfahren die ihren Namen einer Buchstaben Kodierung verdanken.
%QAG noch gegne CQUAD testn erseres müsste schneller sein
\\
In dieser Arbeit wird das sogenannte \textit{CQUAD}-Integrationsverfahren verwendet.
Im Handbuch der GSL wird dieses Verfahren als
\begin{quotation}
    [...]CQUAD is a new doubly-adaptive general-purpose quadrature routine[...]
\end{quotation} \cite*[Kapitel 17.11]{gsl} beschrieben.
\\
Dieses Verfahren kann nicht direkt genutzt werden um die benötigten Integrationen aus \cite*{gasperini:hal-03209144} zu berechnen, da 
der \textit{CQUAD}-Algorithmus zum einen nur für eindimensionale Integrale und nur für den reellen Zahlenraum definiert ist.

Um dies zu ermöglichen sind zwei Anpassungen nötig:
\begin{enumerate}
    \item Das Verschachteln von zwei \textit{CQUAD} Aufrufen, und
    \item das auftrennen in zwei getrennte Aufrufe, jeweils für den Real- und Imaginär-teil der Integranden
\end{enumerate}

Die Implementierung für den Realteil ist in \ref{integration_2d} zu sehen.

\begin{figure}
    \lstinputlisting[language=C++,style=cpp]{gsl_integrator_2d.cpp}
    \caption{Berechnung von komplexem Integral mit GSL}
    \label{integration_2d}
\end{figure}


\subsection{Armadillo}

Armadillo ist eine C++-Bibliothek von \citeauthor{armadillo}, welche Datenstrukturen und Algorithmen der linearen Algebra bereitstellt.
Die Funktionalität uns Syntax ist an die von Matlab angelehnt mit dem Ziel die Umsetzung von \textit{Researchcode} in Produktivcode möglichst einfach zu gestalten \cite{armadillo}.


\begin{center}
    \begin{lstlisting}[language=C++,style=cpp]
        arma::mat A = {{0, 0} , {1, 0}, {0, 1}};
        //use matrix 
        // ...
    \end{lstlisting}
    \begin{lstlisting}[language=C++,style=cpp]
        gsl_matrix * m = gsl_matrix_alloc (3, 2);
        gsl_matrix_set(m, 0, 0) = 0;
        gsl_matrix_set(m, 0, 1) = 0;
        gsl_matrix_set(m, 1, 0) = 1;
        gsl_matrix_set(m, 1, 1) = 0;
        gsl_matrix_set(m, 2, 0) = 0;
        gsl_matrix_set(m, 2, 1) = 1;
        //use matrix 
        // ...
        //free
        gsl_matrix_free (m);
    \end{lstlisting}
    \captionof{figure}{Vergleich von Matrizen in Armadillo und GSL}
    \label{gslarma}
\end{center}

Die GSL bietet zwar auch ein Framework für lineare Algebra, allerdings ist die API von Matlab deutlich moderner und einfacher zu verwenden (siehe Abbildung \ref{gslarma}).

%Warum wurde neben GSL noch Armadilo genutz? => Moderne API von Armadillo

%%Gerade bei der Frage nach dem Framework für Linalg und numerischer Integration lässt sich nicht abschätzen was die beste Lösung ist. Es gibt
%so viele Frameworks und Bibliotheken, dass es nicht möglich ist alle gegeneinander Abzwuwägen.

%Die Entshceidung GSL und Armadillo basier auf: Aramdillo ist einfach einzubinden und unkompliziert in der Anwendung.
%GSL hat eine gute Perfomance. Der Verglecih lief mit Boost wobei einfache Tests zeigten das GSL schneller ist und mit weitaus weniger Aufwand in das Projekt integriert werden kann.
%(Boost integration ist furchtbar!)


\subsection{Intel Threading Building Blocks}


Die Intel\textsuperscript{\textcopyright} oneAPI Threading Building BLocks (oneTBB) ist eine Template-basierte Bibliothek zur effizienten Parallelisierung von Anwendungen. 
Mithilfe von oneTBB ist es möglich mit wenigen Schritten einen Algorithmus auf mehreren Threads auszuführen.
%Die API bietet Datenstrukturen und Algorithmen um di

oneTBB bietet für diverse Parallelisierungsprobleme  Algorithmen und Datenstrukturen an, von einfachen Schleifen, bis hin zu Graph-Based-Parallel Computing.
In dieser Arbeit wurden lediglich die einfacheren Konzepte von parallelen Schleifen genutzt. Für diese Anwendungsfälle stellt oneTBB, unter anderem, die Funktionen \texttt{parallel\_for} und \texttt{parallel\_reduce}
zur Verfügung. Diese arbeiten auf sogenannten \textit{Ranges}, welche eine Abstraktion der zu bearbeitenden Daten sind.

Ranges können die zu verarbeitende Datenmenge in kleinere Bereiche aufteilen. Diese Teilbereiche werden von oneTBB dann parallelisiert. Die Daten innerhalb einer Range werden sequenziell verarbeitet. 
Wie groß diese Teilbereiche werden und welcher Art die Aufteilung ist kann mithilfe von Parametern gesteuert werden. In diser Arbeit werden dies mit der sogenannten Blocked-Ranges (\texttt{blocked\_range}) realisiert, welche die Datenmenge in kontinuierliche Blöcke aufteilt. Die größe der jeweiligen Blöcke wird mit der sogenannten \textit{Grainsize} gesteuert.
oneTBB bietet verschiedene Arten von Blocked-Ranges für bis zu 3 dimensionen. Die Grainsize muss dem zu lösenden Problem entsprechend groß gewählt werden.

\begin{center}
    \lstinputlisting[language=C++,style=cpp]{tbb_example.cpp}
    \captionof{figure}{Minimalbeispiel von \texttt{paralell\_reduce}}
\end{center}

In dieser Arbeit werden ausschließlich Parallelisierungen mithilfe von \texttt{parallel\_recude} umgesetzt.
Dieses bietet die Möglichkeit für jedes Datum eine Transformation durchzuführen und anschließend die tranformierten Daten zu vereinigen. 
So wird beispielsweise in der Implementierung (siehe Abschnitt \ref{gauss_laguerre_section}) des Gauss-Laguerrre Verfahrens ein zu integrierender Pfad parallel an den einzelnen Gauss-Laguerre-Knotenpunkten ausgewertet und entsprechend gewichtet.
Die so berechneten Teilwerte werden nur noch addiert und ergeben so das komplexe (Teil-)Integral des Pfades. 

\subsection{Pybind11}

Das Python-Modul wird mithilfe der Bibliothek \textit{pybind11} umgesetzt.
Pybind11 ist eine sogenannte Header-only Bibliothek mit dem Ziel eine leichtgewichtige Alternative zu bestehenden Bibliotheken zur Erstellung  von
Python-bindings für C++ Bibliotheken anzubieten.
Pybind11\cite{pybind} ist ein Opensource-Projekt und wird unter einer BSD-artigen Lizenz veröffentlicht. 

Folgendes Beispiel\footnote{Beispielcode stammt von \url{https://pybind11.readthedocs.io/en/latest/basics.html} aus der pybind-Dokumenation\cite{pybind}.} stellt die C++Funktion \textit{add} als Python Modul bereit:
\begin{center}
\lstinputlisting[language=C++,style=cpp]{pybind11_example.cpp}
\end{center}
Dieses kann in einer Pythonumgebung wie folgt verwendet werden:

\begin{center}
\begin{lstlisting}[language=Python]
import example;
example.add(1, 2);
\end{lstlisting}
\end{center}



\subsection{Matlab Mex-Funktion}

Um eine C++-Funktion in Matlab zur Verfügung zu Stellen gibt es mehrere Ansätze.
Zum einen kann mithilfe von Matlab C++-Code kompiliert und anschließend genutzt werden. 
Die für den Endnutzer einfacherer Variante ist das Bereitstellen von sogenannten Mex-Funktionen.

Um eine Mex-Funktion zu implementieren müssen die Header-Dateien \texttt{mex.hpp} und \texttt{mexAdapter.hpp} inkludiert werden
und eine Funktion mit dem Namen \textit{MexFunction} implementiert werden. 

In Abbildung \ref{mexfunction} ist eine Beispiel-Implementierung zu sehen 
\begin{center}
    \lstinputlisting[language=C++,style=cpp]{mex_function.cpp}
    \captionof{figure}{Eine einfache Mex-Funktion ohne Logik}
    \label{mexfunction}
\end{center}

Nachdem diese kompiliert wurde lässt sie sich aus Matlab heraus aufrufen.


\subsection{Sonstige}

Als Testframework für automatisierte Unit-Tests wird die Bilbiothek \textit{catch2} verwendet.
Zum Durchführen der automatisierten Benchmarks wird auf die Bibliothek \textit{Benchmark} von Google zurückgegriffen.


%\subsection{Ungenutzte alternativen}


%\begin{itemize}
%    \item Boost numeric
%    \item OpenCL
%    \item OpenMP
%    \item CUDA
%\end{itemize}


\section{Ausgewählte Codestellen}

In diesem Abschnitt werden Details der Implementierung näher beleuchtet. Es werden ausgewählte Codestellen vorgestellt, die wesentliche Funktionen implementieren.

\subsection{Gauss laguerre integration und Cauchy Integral Theoerem}\label{gauss_laguerre_section}

Die Berechnung der Pfadintegrale wird mithilfe des Gauss-Laguerre-Verfahrens realisiert.
Da die Auswertung der Pfadfunktion an den Gauss-Laguerre-Knoten unabhängig voneinander ist wird diese Berechnung mithilfe der Funktion \texttt{parallel\_determinictic\_reduce} der oneTBB API parallisiert.
Diese Funktion verhält sich ähnlich wie \texttt{parallel\_reduce} hat jedoch ein anderes Verhalten hinsichtlich des sogenannten \textit{splittings} der Ranges. 
Die \textit{Grainsize} ist auf 100 Elemente festgelegt, kleinere Blockgrößen haben in Performanztests keine signifikante Beschleunigung ermöglicht und größere haben die Laufzeit negativ beeinflusst.

\begin{center}
    \lstinputlisting[language=C++,style=cpp]{gauss_laguerre.cpp}
    \captionof{figure}{Implementierung der Gauss-Laguerre Quadratur}
    \label{gauss_laguerre_impl}
\end{center}


\subsection{1D Integration Berechnung der Singularität}

In Kapitel 5.1 \cite[12]{gasperini:hal-03209144} (Formel 16) wird die Formel für den eindimenisionalen Fall, im Fall einer Singularität im Interval $[a_1, b_1]$ wie folgt beschrieben:

\begin{equation}
    I(k,y,a,b,) = I(k,y,a,a_1) + I(k,y,a_1, b_1) + I(k,y,b_1,b)
\end{equation}

Die partiellen Integrale in den Intervallen $[a,a_1]$ und  $[b_1,b_1]$ können mithilfe des \textit{steepest-descent}-Verfahrens berechnet werden.
Lediglich das verbleibende Interval muss mit einem klassischen Integrationsverfahren berechnet werden.
Dieses Verfahren muss also fähig sein Singularitäten in einem Integral zu berechnen. Dies ist nicht bei allen Integrationsverfahren gegeben (z.B. dem sogenannten QAG-Verfahren aus der GSL).

und wird in dieser Arbeit wie folgt berechnet:
\begin{center}
    \lstinputlisting[language=C++,style=cpp,mathescape=true]{1d_integration_singularity.cpp}
    \captionof{figure}{Implementierung des 1D-Falls (Singularität)}
\end{center}

Dabei werden die Intervalle ohne Singularität mit dem Gauss-Laguerre-Verfahren (siehe Abbildung \ref{gauss_laguerre_impl}) berechnet und die verbleibende partielle Integration,
welches die Form
\begin{equation*}
    I(k,y,a,b) := \int_{a}^{b}  \frac{e^ikg(x)}{\sqrt{P(x)}} \,dx 
\end{equation*}
hat, mithilfe eines Aufruf des GSL-Integrators berechnet (Analog zum 2D Fall, siehe Abbildung \ref{integration_2d})



\subsection{2D Integration}\label{2dint}

Ebenso wie die 1D-Integration wird die 2D-Integration als sogenanntes \textit{Functor-Objekt}\ref{sec_functor} Implementiert. 
Im Konstruktor werden die invarianten Parameter übergeben und in dem \textit{\(\)-Operator} werden die Dreiecksparameter, der Blickpunkt $r$ und der Richtungsvektor $\theta$ übergeben.

\begin{center}
    \lstinputlisting[language=C++,style=cpp,mathescape=true]{2d_integral.cpp}
    \captionof{figure}{Berechnung des 2D-Integrals}
    \label{2d_integral}
\end{center}

Die Funktion \textit{get\_partial\_integral} ist im Grunde genommen eine Kopie des eindimenisionalen Falles. Dieser wird allerdings aus Performanzgründen als lokale Funkion implementiert.
Die Kosten eines indirekten Aufrufs des eindimenisionalen Falls haben sich in den Benchmarks sehr negativ dargestellt.


\chapter{Analyse und Auswertung}\label{analysis}


\section{Testsystem}

Die Performanztests wurden auf einem Vierkernprozessor Intel i5-4590 mit 8GB Arbeitsspeicher durchgeführt.


\section{Analyse mit Valgrind}

In diesem Abschnit wird auf die Analyse des Projektes mit Valgrind eingegangen. Zunächst werden Valgrind und die daraus benutzten Tools vorgestellt und dann werden einige Ergebnisse präsentiert.
\subsection{Valgrind}

Valgrind ist einerseits ein Framework zum Erzeugen von dynamsichen Analyse Werkzeugen und andereseits eine Sammlung ebensolcher Werkzeuge (siehe \cite{10.1145/1250734.1250746}). 
Valgrind ist offene Software und wird unter der GNU GPL-2 Lizenz veröffentlicht\footnote{\url{https://valgrind.org/}}.
Die Sammlung bietet acht Werkzeuge\footnote{vgl. \url{https://valgrind.org/info/tools.html}} an von denen eines noch den Status eines experimentellen Werkzeugs hat:
\begin{itemize}
  \item Memcheck, ein Werkzeug zum Analysieren von Speicherlecks.
  \item Cachegrind, ein Cache und Branch-prediction Profiler.
  \item Callgrind, ein Profiler der einen Aufrufgraphen erzeugt.
  \item Helgrind und DRD, Werkzeuge um Threading-Fehler zu erkennen.
  \item Massif und DHAT, Heapprofiler für Analysen hinsichtlich speichereffizienter Programme 
  \item BBV, ein Werkzeug Forschung im Bereich der Rechnerarchitektur
\end{itemize}

In dieser Arbeit wurden die Werkzeuge Memcheck, Cachegrind und Callgrind verwendet. Die Ergebnisse der Werkzeuge DRD und Helgrind waren unbrauchbar, da diese nicht mit oneTBB kompatibel waren.

\subsubsection{cachegrind und callgrind}

Cachegrind und Callgrind, entwickelt von \citeauthor{Weidendorfer2004ATS}, sind sogenannte Cacheprofiler. 
Diese Werkzeuge werden verwendet um die Programmstellen mit der größten Laufzeit festzustellen und verschiedene Umstellungen im Programmcode vergleichen zu können.

Cachegrind simuliert wie das zu testende Programm mit der Cache-Hierarchie und dem Branch-Predictor einer virtuellen Maschine interagiert. Die dabei simulierte Maschine ist orientiert and der Architektur moderner Maschinen.
Dabei werden verschiedene Caches simuliert und die Zugriffe darauf hinsichtlich von Misses ausgewertet. In Abbildung \ref{callgrind} ist ein Beispielaufruf zu sehen.
 

\begin{figure}
  \includegraphics{images/callgrind.png}
  \caption{Beispielaufruf von Callgrind}\label{callgrind}
\end{figure}
%\subsubsection{Memcheck}

%Gehört eigentlihc nciht wirklcih hier her
%Screenshot plus was macht das Tool. Wir haben das genutzt um sicherzustellen, dass es keine Memoryleaks gibt.

\subsubsection{Auswerten von callgrind-Ergebnissen}


Mithilfe der Anwendung QCachegrind können die Ergebnisse von Valgrinds cachegrind/callgrind grafisch ausgewertet werden. QCachegrind ist ein Windowsbuild der Opensource Anwendung KCacheGrind.
KCachegrind ist Teil der Werkzeuge aus der Arbeit \cite{Weidendorfer2004ATS}.

\begin{figure}
  \includegraphics[width=\textwidth]{images/qcachegrind.png}
  \caption{Übersicht von QCachegrind}\label{qcachegrind}
\end{figure}

In Abbildung \ref{qcachegrind} ist eine Übersicht einer Auswertung mit QCachegrind zu sehen. In diesem Beispiel wurde mithilfe des Werkzeugs \texttt{callgrind} eine Aufzeichnung des zweidimensionalen Falls 
mit verschiedenen Wellenzahlen und jeweils 40 Stichproben von Beobachtungspunkt und Richtungsektoren für ein festes Dreieck berechnet.
Abbildung \ref{qcachegrind_result} zeigt einen Ausschnitt des Aufrufgraphen vergrößert dar. In diesem lassen sich an Pfeilen ablesen wie oft welcher Programmteil aufgerufen wird und mithilfe einer prozentualen Angabe einsehen wie viel der Laufzeit in diesem Teil verbraucht wird.

\begin{figure}
  \includegraphics[width=\textwidth]{images/qcachegrind_callgraph.png}
  \caption{Aufrufgraph aus QCachegrind}\label{qcachegrind_result}
\end{figure}

Aus diesen und weiteren Auswertungen werden die meistaufgerufene Funktion und der Hotpath ersichtlich.

\section{Ergebnisse der Optimierungsmaßnahmen}

In diesem Abschnitt wird auf einige Ergebnisse der manuellen Performance-Messungen eingegangen.

\subsection{Hotpath}

Der häufigste Fall der Anwendung ist das Berechnen des Integrals in Situationen in denen keine Singularität auftritt. 
Dementsprechend werden diese in den implementierten Funktionen als der regelfall betrachtet.
Durch das vorziehen der Codestellen die diese Szenarien abhandeln werden beispielsweise die benötigten Sprunganweisungen geringer gehalten.



\subsection{Meistaufgerufene Funktion}

Die Funktion mit der größten Laufzeit ist das \textit{Steepest-descent}-Verfahren.
Diese Funktion ist über mehrere Iterationen optimiert worden, bis sie die im Kapitel \ref{Implementierung} dargestellte Form erreichte.
In früheren Versionen war dieses Verfahren in einer eigenen C++-Klasse implementiert, allerdings haben die Auswertungen gezeigt, dass ein
direkterer Aufruf der Gauss-Laguerre-Quadratur einen Laufzeitgewinn von fast 50 Prozent erzielen lies.


\section{Auswertung der Laufzeitmessugen}

Die folgenen Laufzeitvergleiche wurden aus einer Matlab Umgebung heraus ausgeführt, d.h. es werden die ursprüngliche Implementierung sowie das Matlab-Modul der C++ Implementierung verglichen.


\subsection{Iteration über Wellenzahl}


In diesem Test wird die Laufzeit hinsichtlich veränderter Wellenzahl $k \in \{ 100, 500, 1000, 3000, 5000 \}$ gemessen.
Für jedes $k$ werden 50 zufällige Richtungsvektoren $r$ berechnet mit 600 Gauss-Laguerre-Knoten berechnet.

\begin{equation}
  A = \begin{pmatrix}
      0 & 0 \\
      1 & 0 \\
      0 & 1 \\
  \end{pmatrix}, b = \begin{pmatrix}
      0 \\ 0\\ 0
  \end{pmatrix},
\end{equation}

\begin{center}
    \begin{tikzpicture}
        \begin{axis}[
          width=\textwidth,
                  %width=3.358in,
        %height=2.309in,
        %at={(0.563in,0.312in)},
        scale only axis,
        yticklabel style={
          /pgf/number format/fixed,
          /pgf/number format/precision=2,
          /pgf/number format/fixed zerofill
        },
        extra y ticks={-0.05, 0.05},
        scaled y ticks=false,
        xlabel=Wellenzahl k,
        ylabel=\text{CPU-Laufzeit [s]},
        %xmin=0,
       % xmax=6000,
        %ymin=0,
        %ymax=0.5,
        axis background/.style={fill=white},
        legend style={legend cell align=left, align=left, draw=white!15!black}
        ]
        \addplot+[
      green, mark options={green, scale=0.75},
      smooth, 
      error bars/.cd, 
        y fixed,
        y dir=both, 
        y explicit ] table [color=green, mark=o, mark options={solid, green} x=k, y=t,y error=std, col sep=comma] { 
            k, t, std 
            100, 0.0546875000000000, 0.0377819439281490
            500, 0.0421875000000000, 0.0304832595619367
            1000, 0.0396875000000000, 0.0304767208961468
            3000, 0.0393750000000000, 0.0399597214296433
            5000, 0.0340625000000000, 0.0374428241844571         
          };
          \addplot+[
            red, mark options={red, scale=0.75},
            smooth, 
            error bars/.cd, 
              y fixed,
              y dir=both, 
              y explicit ] table [color=red, mark=o, mark options={solid, red} x=k, y=t,y error=std, col sep=comma] { 
                  k, t, std 
                  100, 0.120625000000000, 0.0825011595465820
                  500, 0.0928125000000000, 0.0439012055439740
                  1000, 0.0959375000000000, 0.0326549607327770
                  3000, 0.120625000000000, 0.164181352722021
                  5000, 0.0943750000000000, 0.0724645849462240          
                };
        %\addplot [color=red, draw=none, mark=o, mark options={solid, mycolor1}]
        %  table[row sep=crcr] file {..\data\performance_matlab.csv};
        \addlegendentry{C++ Implementierung}
        \addlegendentry{MatLab Implementierung}

        \end{axis}
    \end{tikzpicture}%
    \captionof{figure}{Laufzeitvergleich über verschiedene Wellenzahlen $k$}
\end{center}


\subsection{Laufzeitvergleich bei steigender Auflösung}


In diesem Test wird die Laufzeit hinsichtlich veränderter Auflösung $res \in \{ 0.1, 0.01, 0.001, 0.0001 \}$ gemessen.
Für jedes $k$ werden 50 zufällige Richtungsvektoren $r$ berechnet mit 600 Gauss-Laguerre-Knoten berechnet.

\begin{equation}
  A = \begin{pmatrix}
      0 & 0 \\
      1 & 0 \\
      0 & 1 \\
  \end{pmatrix}, b = \begin{pmatrix}
      0 \\ 0\\ 0
  \end{pmatrix},
\end{equation}
    
\begin{center}
    \begin{tikzpicture}
        \begin{axis}[
        %width=3.358in,
        %height=2.309in,
        %at={(0.563in,0.312in)},
        scale only axis,
        xmode=log,
        ymode=log,
        ymax=1000,
        width=\textwidth,
        %log ticks with fixed point,
        x dir=reverse,
        ylabel=\text{CPU-Laufzeit [s] (log)},
        xlabel=Layer-Auflösung (log),
        % for log axes, x filter operates on LOGS.
        % and log(x * 1000) = log(x) + log(1000):
        %x filter/.code=\pgfmathparse{#1 + 6.90775527898214},
        axis background/.style={fill=white},
        legend style={legend cell align=left, align=left, draw=white!15!black}
        ]
        \addplot+[
      green, mark options={green, scale=0.75},
      smooth, 
      error bars/.cd, 
        y fixed,
        y dir=both, 
        y explicit ] table [color=green, mark=o, mark options={solid, green} x=res, y=t,y error=std, col sep=comma] { 
            res, t, std
            0.1000,    0.0563,    0.0461
            0.0100,    0.3469,    0.0645
            0.0010,   3.8547,    1.6996
            0.0001,  38.9781,   15.6876
          };
          \addplot+[
            red, mark options={red, scale=0.75},
            smooth, 
            error bars/.cd, 
              y fixed,
              y dir=both, 
              y explicit ] table [color=red, mark=o, mark options={solid, red} x=res, y=t,y error=std, col sep=comma] { 
                  res, t, std
                  0.1000,    0.1375,    0.0799
                  0.0100,    0.8688,    0.0834
                  0.0010,    9.5406,    4.2787
                  0.0001,   86.2375,   27.7633
                };
        %\addplot [color=red, draw=none, mark=o, mark options={solid, mycolor1}]
        %  table[row sep=crcr] file {..\data\performance_matlab.csv};
        \addlegendentry{C++ Implementierung}
        \addlegendentry{MatLab Implementierung}
        \end{axis}
    \end{tikzpicture}%
    \captionof{figure}{Laufzeitvergleich über verschiedene Auflösungen, logarithmische Skalen}
\end{center}


\section{Vergleiche der Genauigkeit}

Auf das Fehlerszenarion eingehen.
Graph mit dem Fehlervergleich

Daten des relativen Fehlers messen 
=> matlab um die 7-e18 
=> gsl um die 1-e18

Das einfach mal auf einem Lauf rennen lassen


\chapter{Zusammenfassung und Ausblick}

Insgesamt hat diese Arbeit die geforderten Anforderungen hinsichtlich der Laufzeit und Genauigkeit erfüllt. Die Implementierung ist hinsichtlich der Performanz besser als die zugrunde liegende Matlab-Implementierung.
Durch das Python-Modul und die Matlab-Funktionen ist es relativ einfach möglich die Bibliothek zu verwenden. 
Die Implementierung hat sich als komplizierter als erwartet gezeigt, denn einige Details werden in der Arbeit \cite{gasperini:hal-03209144} nicht näher beschrieben, die in der Implementierung nicht trivial sind.

So ist beispielsweise die Berechnung der im Programmcode \textit{Splitting-point} genannten Punkte nicht weiter beschrieben und war nur mithilfe des Matlab-Codes nachvollziehbar.

Beim Auswerten der numerischen Genauigkeiten sind einige Probleme aufgetreten, welche sich später auf Implementierungsfehler zurückführen liessen. Auch ist bei diesen Tests aufgefallen,
wie rapide die Laufzeit der Berechnung des Integrals über Dreiecksoberflächen mit der Matlab \textit{integral2}-Funktion mit steigender Wellenzahl ansteigt.
Ein Aufgrund des vorhin erwähnten Implementierungsfehlers ein unbrauchbarer Test über Wellenzahlen $k \in \{1000, 3000, 5000\}$ wurde nach 8 Stunden abgebrochen, ohne dass die Berechnungen abgeschlossen waren.
Die Berechnung mit dem \textit{Steepest-descent}-Verfahren lief dahingegen innerhalb weniger Minuten durch.
Hinsichtlich der Perfomanz bestehen noch einige Möglichkeiten die in dieser Arbeit nicht ausgeschöpft wurden. Insbesondere könnte eine Beschleunigung der Verfahren mithilfe der GPU enorme Laufzeit-Verbesserungen mit sich bringen.
%Des weiteren könnte eine andere Herangehensweise hinsichtlich der gewählten Architektur womöglich weitere Lücken schließen. Die hier gewählte Lösung folg hinsichtlich der einzelnen Berechnungsschritte des Verfahrens der mathematischen Formulierung.
%Dahingegen könnte eine Architektur die auf größere Mengen von Eingabeparemetern 

%\section{Schwierigkeiten}

%Komplizierte Mathematik dahinter, 
%speziell die Auswahl der Punkte um die Splitting Points konnte ich nicht nachvollziehen. (Keine explizierte Erklärung im Paper, dafür aber im MatlabCode)


%\section{Weitere Möglichkeiten das Verfahren zu beschleunigen}
%mögliche beschlenigungen die nicht bearbeitet wurden:
%\begin{itemize}
 %   \item Detailliertes Auseinandersetzen mit verschiedenen Complier und Linker optionen
 %   \item Tooling angepasst an Zielarchitektur (?)
 %   \item Pipelinebasierte Architektur (siehe TBB dokumente!)
 %   \item GPU beschleunigung: steht und fällt mit numerischer Integration auf GPU (Eigene implementierung von Numerischer Integration würde viele Möglichkeiten bieten)
 %   \item Templated/Constexpression zeug: Je nach Szenario könnte eine Compiletime optimierte Anwendung bereitgestellt werden. So könnten zum einen Parameter wie die Wellenfrequenz $k$ bereits in das Programm kompliiert werden un so zu einer enomren beschleunigugn beitragen.
 %   \item Templated/Constexpression 2: Für fixe Node größen der Laguerre Implementeirung könnt eine gezielte festgelegte Implementierung durch eine fixe FUnktion zur Kompilezeit ausgetauscht werden und so für eine weitere beschleunigung sorgen?
%\end{itemize}


%\secion{Fazit}


%In diesem Kapitel soll die Arbeit noch einmal kurz zusammengefasst werden. Insbesondere sollen die wesentlichen Ergebnisse Ihrer Arbeit herausgehoben werden. Erfahrungen, die z.B. Benutzer mit der Mensch-Maschine-Schnittstelle gemacht haben oder Ergebnisse von Leistungsmessungen sollen an dieser Stelle präsentiert werden. Sie können in diesem Kapitel auch die Ergebnisse oder das Arbeitsumfeld Ihrer Arbeit kritisch bewerten. Wünschenswerte Erweiterungen sollen als Hinweise auf weiterführende Arbeiten erwähnt werden.



%------------------ Literaturverzeichnis & Index -------------------------------
\backmatter
\printbibliography
%\bibliography{literatur}								% Literaturverzeichnis (literatur.bib)
\printindex												% Index (optional)


%------------------ Anhänge ----------------------------------------------------
\begin{appendix}
	\include{chapters/Glossar}							% Glossar (optional)
	\include{chapters/Selbststaendigkeitserklaerung}	% Selbstständigkeitserklärung
\end{appendix}


\end{document}
