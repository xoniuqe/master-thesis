\chapter{Das Verfahren}\label{algo}

\section{Accoustic single-layer integral operator}

In der Arbeit \citetitle{gasperini:hal-03209144} wird ein Verfahren zum lösen von Integralen der Form
\begin{equation}
    \Delta
    I_{r,\Delta}(k) = \int_{\Delta}^{}  \frac{e^{ik(\left\lVert r-r'\right\rVert + \theta \cdot r')}}{\left\lVert r-r'\right\rVert} \,dr'
\end{equation}
vorgestellt. Das Integral $\I_{r,\Delta}(k)$ ist das Integral eines Dreicks $\delta$ mit Beobachtungspunkt $r \in \mathbb{R}^3$ für die Wellenzahl $k$. $\theta \in \mathbb{R}^3$ ist ein Richtungsvektor.
Dieses Integral wird auch als \textit{accoustic single-layer Integral-operator} bezeichnet.
Mit größer werdender Wellenzahl $k$ wird dieses Problem als im \textit{high frequency regime} liegend beschrieben und die Berechnung dieser ist mit einer hohen Laufzeit verbunden.
Die Autoren füren aus, dass sich das Integral so berechnen lässt, dass das Integral mithilfe des Cauchy-Integral-Theorems gezielt an solchen Pfaden berechnen lässt, die nicht mit $k$ oszilieren.
Dies ermöglicht es diese Pfade mit einem \textit{steepest-descent}-Verfahren effizient zu berechnen.

Sie führen weiter aus wie diese Pfade gefunden werden können damit das Integral wie folgt umgestellt werden kann \footnote{\cite[Kaptiel 1, Gleichung 4]{gasperini:hal-03209144}}:
\begin{equation}
    \begin{aligned}
        I(k,y,b) = & \int_{h_a}^{} f(z) e^{ikg(z)} \,dz  \int_{h_b}^{} f(z) e^{ikg(z)} \,dz \\
         = & \int_{0}^{\infty} h'_a(t)f(h_a(t))e^{ikg(h_a(t))} \,dt -\int_{0}^{\infty} h'_b(t)f(h_b(t))e^{ikg(h_b(t))} \,dt 
    \end{aligned}
\end{equation}

Dabei sind $h_a$ und $h_b$ die Pfade gemäßt des Cauchy-Integral-Theorems. %Ein Teilintegral $\int_{0}^{\infty} h'_a(t)f(h_a(t))e^{ikg(h_a(t))} \,dt$ kann dann mithilfe des Gauss-Laguerre-Verfahrens 

Die Autoren teilen dieses Problem in zwei Fälle auf, einen eindimensinalen Fall und einen zweidimensionalen Fall der die Berechnung eines Integrals über ein Dreiecks abdeckt.


\subsection{Fall1: Das 1D-Verfahren}

In der Arbeit \cite{gasperini:hal-03209144} wird zunächst beschrieben wie sich das Problem auf Pfaden in der komplexen Ebene berechnen lässt auf denen $e^{ikg(x)}$ nicht oszilliert.
Es wird gezeigt wie sogenannte \textit{Splitting-points} berechnet werden, welche als Endpunkte für diese Pfade dienen.
Liegt einer dieser Punkte auf einem komplexen Pfad, so kann das \textit{steepest-descent}-Verfahren nicht angewandt werden und es muss auf ein klassisches Integrationsverfahren zurückgegriffen werden, welches diese Singulariäten berechnen kann.

Der eindimenisionale Fall wird in Kapitel 5.1 von \cite{gasperini:hal-03209144} beschrieben: 

\begin{equation}
    I(k,y,a,b,) = I(k,y,a,a_1) + I(k,y,a_1, b_1) + I(k,y,b_1,b)
\end{equation}

Dabei enthält das Intervall $[a_1,b_1]$ einen sogenannten Splitting-point und das Integral $I(k,y,a_1,b_1)$ muss mit einem klassischen Verfahren berechnet werden.


\subsection{Das Verfahren über Dreiecksoberflächen}

In Kapitel 5.2 von \cite*{gasperini:hal-03209144} wird gezeigt wie das Verfahren auf das ursprüngliche Ziel erweitert wird.
Dazu wird das zu berechnende Integral für ein Dreieck in mehrere parallele Schichten aufgeteilt, welche jeweils mit dem eindimenisionalen Verfahren angehnähert werden können.


\begin{equation}
    \begin{aligned}
        \hat{I}_{\Delta_1,n_y}(k) = & \sum_{j = 1}^{n_y} \left[ \Lambda_0(y_{m_j}) \int_{y_{j-1}}^{y_j}  e^{ik(\sqrt(P(0,y))+q_yy+s)}\,dy \right.\\
        & \left.  - \Lambda_{1-y_{m_j}}(y_{m_j}) \int_{y_{j-1}}^{y_j}  e^{ik(\sqrt(P(1-y,y))+q_x(1-y)+q_yy+s)}\,dy  \right] \\
    \end{aligned}
\end{equation}

Kjaldkfj



\section{Mathematische Grundlagen}


\subsection{Cauchyscher Integralsatz}




\subsection{Gauss-Laguerre Quadratur}

Die Gauss-Laguerre Quadratur ist eine Approxmiation zur Auswertung von Integralen:
\begin{equation}
    \int_{0}^{\inf } f(t) e^-t  \,dt  \approx \sum_{i = 1}^{n} w_if(x_i) 
\end{equation}

Diese wird in dem steepest-descent Verfahren in Kombination mit dem Cauchy-Integral-Theorem genutzt um das Integral "LINK AUF FORMEL" zu berechnen.
