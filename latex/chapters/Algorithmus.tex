\chapter{Das Verfahren}

In diesem Kapitel wird das zu implemnetierende VErfahren vorgestellt.
\section{Accoustic single-layer integral operator}

Welches Problem löst das Verfahren:
\begin{equation}
    \Delta
    I_{r,\Delta}(k) = \int_{\Delta}^{}  \frac{e^{ik(\left\lVert r-r'\right\rVert + \theta \cdot r')}}{\left\lVert r-r'\right\rVert} \,dr'
\end{equation}

Wie wird es im Paper angegangen:


\begin{equation} \label{eq:2}
    \Delta
    I_{r,\Delta}(k) = \int_{\Delta}^{}  \frac{e^{ik(\left\lVert r-r'\right\rVert + \theta \cdot r')}}{\left\lVert r-r'\right\rVert} \,dr'
\end{equation}

Und dann auf die relevanten Mathematischen "Grundlagen" eingehen die verwendet werden um das Verfahren implementieren:

\section{Mathematische Grundlagen}
\subsection{Cauchy-Integral-Theorem}




\subsection{Gauss-Laguerre Quadratur}

Die Gauss-Laguerre Quadratur ist eine Aproxmiation zur Auswertung von Integralen:
\begin{equation}
    \int_{0}^{\inf } f(t) e^-t  \,dt  \approx \sum_{i = 1}^{n} w_if(x_i) 
\end{equation}

Diese wird in dem steepest-descent Verfahren in Kombination mit dem Cauchy-Integral-Theorem genutzt um das Integral "LINK AUF FORMEL" zu berechnen.

\subsection{Steepest Descent Method}

In der Arbeit von \cite{gasperini:hal-03209144} wird das Integral \ref{eq:2} aus Formel 1 für ein Dreieck $\Delta \in \mathbb{R}^3$ und einen Einheitsvektor $\theta \in \mathbb{R}^3$ wie folgt definiert:
\begin{equation}
    I_{\Delta}(k) = | A_1 \times A_2 | \int_{0}^{1} \int_{0}^{1-y} \frac{e^{ik(\left\lVert Ax + b - r\right\rVert+\theta \cdot(Ax +b))}}{\left\lVert Ax + b - r\right\rVert }  \,dx \,dy 
\end{equation}
Dabei wird das Dreieck $\Delta$ als parametrisierties Einheitsdreieck mit der affinen Abbildung $x \mapsto Ax + $ mit $x = (x, y)^T, A \in \mathbb{R}^3$ und $b \in \mathbb{R}^3$ beschrieben.

Zusammen mit der Wellenzahl $k$ sind $A$, $b$, $r$ und $\theta$ die Eingabegrößen des zu implementierenden Algorithmus.
In Kapitel 2 von \cite*{gasperini:hal-03209144} wird hergeleitet wie die Gleichung \ref{2} in Abhängigkeit von diesen Eingabegrößen
definieren lässt.
Die Funktion $g$ in Gleichung \ref{2} lässt sich mithilfe der Funktion $P(x)$: 
\begin{equation}
    P(x) = c_0 x^2 + c_0|c|^2-2c_0xRe(c)
\end{equation}
wie folgt beschreiben:
\begin{equation}
    g(x) = \sqrt{P(x)} + qx + s
\end{equation}

 
