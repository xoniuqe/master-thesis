\chapter{Das Verfahren}

\section{Accoustic single-layer integral operator}

In der Arbeit \citetitle{gasperini:hal-03209144} wird ein Verfahren zum lösen von Problemen der folgenden Art
\begin{equation}
    \Delta
    I_{r,\Delta}(k) = \int_{\Delta}^{}  \frac{e^{ik(\left\lVert r-r'\right\rVert + \theta \cdot r')}}{\left\lVert r-r'\right\rVert} \,dr'
\end{equation}
vorgestellt.

Die Autoren zeigen, dass sich das Problem wie folgt umformulieren lässt:

\begin{equation}
    I(k,a,b) = \int_{0}^{\infty} h'_a(t)f(h_a(t))e^{ikg(h_a(t))} \,dt - \int_{0}^{\infty} h'_b(t)f(h_b(t))e^{ikg(h_b(t))} \,dt
\end{equation}


Und dann auf die relevanten Mathematischen "Grundlagen" eingehen die verwendet werden um das Verfahren implementieren:

Idee: Die einzelnen "teile" des Algorithmuses getrennt erklären:
Fall mit und ohne Singularität und dann in der Implementierung darauf referenzieren und die jeweiligen Teile verlinken.

\subsection{Das 1D-Verfahren}
\subsection{Das Verfahren über Dreiecksoberflächen}

In Kapitel 5.2 von \cite*{gasperini:hal-03209144} wird gezeigt wie das Verfahren auf das ursprüngliche Ziel erweitert wird.
Dazu wird das zu berechnende Integral für ein Dreieck in mehrere parallele Schichten aufgeteilt, welche jeweils mit dem eindimenisionalen Verfahren angehnähert werden können.

\begin{equation}
    \resizebox{.9\hsize}{!}{$\hat{I}_{\Delta_1,n_y}(k) = \sum_{j = 1}^{n_y} \left[ \Lambda_0(y_{m_j}) \int_{y_{j-1}}^{y_j}  e^{ik(\sqrt(P(0,y))+q_yy+s)}\,dy - \Lambda_{1-y_{m_j}}(y_{m_j}) \int_{y_{j-1}}^{y_j}  e^{ik(\sqrt(P(1-y,y))+q_x(1-y)+q_yy+s)}\,dy  \right]$}
\end{equation}

Die Anzahl $n_y$ dieser Schichten wird als Parameter der zu implementierenden Lösung angesehen.




\section{Mathematische Grundlagen}
\subsection{Cauchy-Integral-Theorem}




\subsection{Gauss-Laguerre Quadratur}

Die Gauss-Laguerre Quadratur ist eine Aproxmiation zur Auswertung von Integralen:
\begin{equation}
    \int_{0}^{\inf } f(t) e^-t  \,dt  \approx \sum_{i = 1}^{n} w_if(x_i) 
\end{equation}

Diese wird in dem steepest-descent Verfahren in Kombination mit dem Cauchy-Integral-Theorem genutzt um das Integral "LINK AUF FORMEL" zu berechnen.

\subsection{Steepest Descent Method}

In der Arbeit von \cite{gasperini:hal-03209144} wird das Integral \ref{eq:2} aus Formel 1 für ein Dreieck $\Delta \in \mathbb{R}^3$ und einen Einheitsvektor $\theta \in \mathbb{R}^3$ wie folgt definiert:
\begin{equation}
    I_{\Delta}(k) = | A_1 \times A_2 | \int_{0}^{1} \int_{0}^{1-y} \frac{e^{ik(\left\lVert Ax + b - r\right\rVert+\theta \cdot(Ax +b))}}{\left\lVert Ax + b - r\right\rVert }  \,dx \,dy 
\end{equation}
Dabei wird das Dreieck $\Delta$ als parametrisierties Einheitsdreieck mit der affinen Abbildung $x \mapsto Ax + $ mit $x = (x, y)^T, A \in \mathbb{R}^3$ und $b \in \mathbb{R}^3$ beschrieben.

Zusammen mit der Wellenzahl $k$ sind $A$, $b$, $r$ und $\theta$ die Eingabegrößen des zu implementierenden Algorithmus.
In Kapitel 2 von \cite*{gasperini:hal-03209144} wird hergeleitet wie die Gleichung \ref{2} in Abhängigkeit von diesen Eingabegrößen
definieren lässt.
Die Funktion $g$ in Gleichung \ref{2} lässt sich mithilfe der Funktion $P(x)$: 
\begin{equation}
    P(x) = c_0 x^2 + c_0|c|^2-2c_0xRe(c)
\end{equation}
wie folgt beschreiben:
\begin{equation}
    g(x) = \sqrt{P(x)} + qx + s
\end{equation}

 Worauf will ich hinaus?


