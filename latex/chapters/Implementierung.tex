\chapter{Implementierung}\label{impl}

In diesem Kapitel wird dargelegt wie der Algorithmus implementiert wird. Zu Beginn wird erläutert warum welche Technologie zum Einsatz kommt und welche Alternativen es gibt.
\section{Verwendete Technologien}

In diesem Abschnit werden die verwendeten Technologien vorgestellt. 


\subsection{GNU Scientific Library}

Die GNU Scientific Library (GSL) ist eine Sammlung von numerischen Funktionen\cite{gsl}. Die GSL wird unter der GNU General Public License veröffentlicht und ist in der Programmiersprache C geschrieben.

Die GSL bietet unter anderem Funktionen für \textit{BAsic linear Algebrar Subprograms}(BLAS), verschiedene Interpolationsalgorihtmen, Monte-Carlo-Algorithmen, Implementierung für Fast Forier Tranformationen und die für diese 
Arbeit benötigten Algorithmen für die numerische Integration.
GSL bietet eine Reimplementierung des QUADPACK\cite{quadpack}, eine Fortran Bibiliothek für numerische Integration.
Diese umfasst eine mehrere verschiedene Verfahren die ihren Namen einer Buchstaben Kodierung verdanken.
%QAG noch gegne CQUAD testn erseres müsste schneller sein
\\
In dieser Arbeit wird das sogenannte \textit{CQUAD}-Integrationsverfahren verwendet.
Im Handbuch der GSL wird dieses Verfahren als
\begin{quotation}
    [...]CQUAD is a new doubly-adaptive general-purpose quadrature routine[...]
\end{quotation} \cite*[Kapitel 17.11]{gsl} beschrieben.
\\
Dieses Verfahren kann nicht direkt genutzt werden um die benötigten Integrationen aus \cite*{gasperini:hal-03209144} zu berechnen, da 
der \textit{CQUAD}-Algorithmus zum einen nur für eindimensionale Integrale und nur für den reellen Zahlenraum definiert ist.

Um dies zu ermöglichen sind zwei Anpassungen nötig:
\begin{enumerate}
    \item Das Verschachteln von zwei \textit{CQUAD} Aufrufen, und
    \item das auftrennen in zwei getrennte Aufrufe, jeweils für den Real- und Imaginär-teil der Integranden
\end{enumerate}

Die Implementierung für den Realteil ist in \ref{integration_2d} zu sehen.

\begin{figure}
    \lstinputlisting[language=C++,style=cpp]{gsl_integrator_2d.cpp}
    \caption{Berechnung von komplexem Integral mit GSL}
    \label{integration_2d}
\end{figure}


\subsection{Armadillo}

Armadillo ist eine C++-Bibliothek von \citeauthor{armadillo}, welche Datenstrukturen und Algorithmen der linearen Algebra bereitstellt.
Die Funktionalität uns Syntax ist an die von Matlab angelehnt mit dem Ziel die Umsetzung von \textit{Researchcode} in Produktivcode möglichst einfach zu gestalten \cite{armadillo}.


\begin{center}
    \begin{lstlisting}[language=C++,style=cpp]
        arma::mat A = {{0, 0} , {1, 0}, {0, 1}};
        //use matrix 
        // ...
    \end{lstlisting}
    \begin{lstlisting}[language=C++,style=cpp]
        gsl_matrix * m = gsl_matrix_alloc (3, 2);
        gsl_matrix_set(m, 0, 0) = 0;
        gsl_matrix_set(m, 0, 1) = 0;
        gsl_matrix_set(m, 1, 0) = 1;
        gsl_matrix_set(m, 1, 1) = 0;
        gsl_matrix_set(m, 2, 0) = 0;
        gsl_matrix_set(m, 2, 1) = 1;
        //use matrix 
        // ...
        //free
        gsl_matrix_free (m);
    \end{lstlisting}
    \captionof{figure}{Vergleich von Matrizen in Armadillo und GSL}
    \label{gslarma}
\end{center}

Die GSL bietet zwar auch ein Framework für lineare Algebra, allerdings ist die API von Matlab deutlich moderner und einfacher zu verwenden (siehe Abbildung \ref{gslarma}).

%Warum wurde neben GSL noch Armadilo genutz? => Moderne API von Armadillo

%%Gerade bei der Frage nach dem Framework für Linalg und numerischer Integration lässt sich nicht abschätzen was die beste Lösung ist. Es gibt
%so viele Frameworks und Bibliotheken, dass es nicht möglich ist alle gegeneinander Abzwuwägen.

%Die Entshceidung GSL und Armadillo basier auf: Aramdillo ist einfach einzubinden und unkompliziert in der Anwendung.
%GSL hat eine gute Perfomance. Der Verglecih lief mit Boost wobei einfache Tests zeigten das GSL schneller ist und mit weitaus weniger Aufwand in das Projekt integriert werden kann.
%(Boost integration ist furchtbar!)


\subsection{Intel Threading Building Blocks}


Die Intel\textsuperscript{\textcopyright} oneAPI Threading Building BLocks (oneTBB) ist eine Template-basierte Bibliothek zur effizienten Parallelisierung von Anwendungen. 
Mithilfe von oneTBB ist es möglich mit wenigen Schritten einen Algorithmus auf mehreren Threads auszuführen.
%Die API bietet Datenstrukturen und Algorithmen um di

oneTBB bietet für diverse Parallelisierungsprobleme  Algorithmen und Datenstrukturen an, von einfachen Schleifen, bis hin zu Graph-Based-Parallel Computing.
In dieser Arbeit wurden lediglich die einfacheren Konzepte von parallelen Schleifen genutzt. Für diese Anwendungsfälle stellt oneTBB, unter anderem, die Funktionen \texttt{parallel\_for} und \texttt{parallel\_reduce}
zur Verfügung. Diese arbeiten auf sogenannten \textit{Ranges}, welche eine Abstraktion der zu bearbeitenden Daten sind.

Ranges können die zu verarbeitende Datenmenge in kleinere Bereiche aufteilen. Diese Teilbereiche werden von oneTBB dann parallelisiert. Die Daten innerhalb einer Range werden sequenziell verarbeitet. 
Wie groß diese Teilbereiche werden und welcher Art die Aufteilung ist kann mithilfe von Parametern gesteuert werden. In diser Arbeit werden dies mit der sogenannten Blocked-Ranges (\texttt{blocked\_range}) realisiert, welche die Datenmenge in kontinuierliche Blöcke aufteilt. Die größe der jeweiligen Blöcke wird mit der sogenannten \textit{Grainsize} gesteuert.
oneTBB bietet verschiedene Arten von Blocked-Ranges für bis zu 3 dimensionen. Die Grainsize muss dem zu lösenden Problem entsprechend groß gewählt werden.

\begin{center}
    \lstinputlisting[language=C++,style=cpp]{tbb_example.cpp}
    \captionof{figure}{Minimalbeispiel von \texttt{paralell\_reduce}}
\end{center}

In dieser Arbeit werden ausschließlich Parallelisierungen mithilfe von \texttt{parallel\_recude} umgesetzt.
Dieses bietet die Möglichkeit für jedes Datum eine Transformation durchzuführen und anschließend die tranformierten Daten zu vereinigen. 
So wird beispielsweise in der Implementierung (siehe Abschnitt \ref{gauss_laguerre_section}) des Gauss-Laguerrre Verfahrens ein zu integrierender Pfad parallel an den einzelnen Gauss-Laguerre-Knotenpunkten ausgewertet und entsprechend gewichtet.
Die so berechneten Teilwerte werden nur noch addiert und ergeben so das komplexe (Teil-)Integral des Pfades. 

\subsection{Pybind11}

Das Python-Modul wird mithilfe der Bibliothek \textit{pybind11} umgesetzt.
Pybind11 ist eine sogenannte Header-only Bibliothek mit dem Ziel eine leichtgewichtige Alternative zu bestehenden Bibliotheken zur Erstellung  von
Python-bindings für C++ Bibliotheken anzubieten.
Pybind11\cite{pybind} ist ein Opensource-Projekt und wird unter einer BSD-artigen Lizenz veröffentlicht. 

Folgendes Beispiel\footnote{Beispielcode stammt von \url{https://pybind11.readthedocs.io/en/latest/basics.html} aus der pybind-Dokumenation\cite{pybind}.} stellt die C++Funktion \textit{add} als Python Modul bereit:
\begin{center}
\lstinputlisting[language=C++,style=cpp]{pybind11_example.cpp}
\end{center}
Dieses kann in einer Pythonumgebung wie folgt verwendet werden:

\begin{center}
\begin{lstlisting}[language=Python]
import example;
example.add(1, 2);
\end{lstlisting}
\end{center}



\subsection{Matlab Mex-Funktion}

Um eine C++-Funktion in Matlab zur Verfügung zu Stellen gibt es mehrere Ansätze.
Zum einen kann mithilfe von Matlab C++-Code kompiliert und anschließend genutzt werden. 
Die für den Endnutzer einfacherer Variante ist das Bereitstellen von sogenannten Mex-Funktionen.

Um eine Mex-Funktion zu implementieren müssen die Header-Dateien \texttt{mex.hpp} und \texttt{mexAdapter.hpp} inkludiert werden
und eine Funktion mit dem Namen \textit{MexFunction} implementiert werden. 

In Abbildung \ref{mexfunction} ist eine Beispiel-Implementierung zu sehen 
\begin{center}
    \lstinputlisting[language=C++,style=cpp]{mex_function.cpp}
    \captionof{figure}{Eine einfache Mex-Funktion ohne Logik}
    \label{mexfunction}
\end{center}

Nachdem diese kompiliert wurde lässt sie sich aus Matlab heraus aufrufen.


\subsection{Sonstige}

Als Testframework für automatisierte Unit-Tests wird die Bilbiothek \textit{catch2} verwendet.
Zum Durchführen der automatisierten Benchmarks wird auf die Bibliothek \textit{Benchmark} von Google zurückgegriffen.


%\subsection{Ungenutzte alternativen}


%\begin{itemize}
%    \item Boost numeric
%    \item OpenCL
%    \item OpenMP
%    \item CUDA
%\end{itemize}


\section{Ausgewählte Codestellen}

In diesem Abschnitt werden Details der Implementierung näher beleuchtet. Es werden ausgewählte Codestellen vorgestellt, die wesentliche Funktionen implementieren.

\subsection{Gauss laguerre integration und Cauchy Integral Theoerem}\label{gauss_laguerre_section}

Die Berechnung der Pfadintegrale wird mithilfe des Gauss-Laguerre-Verfahrens realisiert.
Da die Auswertung der Pfadfunktion an den Gauss-Laguerre-Knoten unabhängig voneinander ist wird diese Berechnung mithilfe der Funktion \texttt{parallel\_determinictic\_reduce} der oneTBB API parallisiert.
Diese Funktion verhält sich ähnlich wie \texttt{parallel\_reduce} hat jedoch ein anderes Verhalten hinsichtlich des sogenannten \textit{splittings} der Ranges. 
Die \textit{Grainsize} ist auf 100 Elemente festgelegt, kleinere Blockgrößen haben in Performanztests keine signifikante Beschleunigung ermöglicht und größere haben die Laufzeit negativ beeinflusst.

\begin{center}
    \lstinputlisting[language=C++,style=cpp]{gauss_laguerre.cpp}
    \captionof{figure}{Implementierung der Gauss-Laguerre Quadratur}
    \label{gauss_laguerre_impl}
\end{center}


\subsection{1D Integration Berechnung der Singularität}

In Kapitel 5.1 \cite[12]{gasperini:hal-03209144} (Formel 16) wird die Formel für den eindimenisionalen Fall, im Fall einer Singularität im Interval $[a_1, b_1]$ wie folgt beschrieben:

\begin{equation}
    I(k,y,a,b,) = I(k,y,a,a_1) + I(k,y,a_1, b_1) + I(k,y,b_1,b)
\end{equation}

Die partiellen Integrale in den Intervallen $[a,a_1]$ und  $[b_1,b_1]$ können mithilfe des \textit{steepest-descent}-Verfahrens berechnet werden.
Lediglich das verbleibende Interval muss mit einem klassischen Integrationsverfahren berechnet werden.
Dieses Verfahren muss also fähig sein Singularitäten in einem Integral zu berechnen. Dies ist nicht bei allen Integrationsverfahren gegeben (z.B. dem sogenannten QAG-Verfahren aus der GSL).

und wird in dieser Arbeit wie folgt berechnet:
\begin{center}
    \lstinputlisting[language=C++,style=cpp,mathescape=true]{1d_integration_singularity.cpp}
    \captionof{figure}{Implementierung des 1D-Falls (Singularität)}
\end{center}

Dabei werden die Intervalle ohne Singularität mit dem Gauss-Laguerre-Verfahren (siehe Abbildung \ref{gauss_laguerre_impl}) berechnet und die verbleibende partielle Integration,
welches die Form
\begin{equation*}
    I(k,y,a,b) := \int_{a}^{b}  \frac{e^ikg(x)}{\sqrt{P(x)}} \,dx 
\end{equation*}
hat, mithilfe eines Aufruf des GSL-Integrators berechnet (Analog zum 2D Fall, siehe Abbildung \ref{integration_2d})



\subsection{2D Integration}\label{2dint}

Ebenso wie die 1D-Integration wird die 2D-Integration als sogenanntes \textit{Functor-Objekt}\ref{sec_functor} implementiert (siehe Abbildung \ref{2d_integral_functor}). 
Im Konstruktor werden die invarianten Parameter übergeben und in dem \textit{\(\)-Operator} werden die Dreiecksparameter, der Blickpunkt $r$ und der Richtungsvektor $\theta$ übergeben.

In diesem Abschnitt wird primär auf die Berechnung eines einzelnen Schicht $n$ eingangen. Die Anzahl der zu berechnenden Schichten wird aus der Konfiguration(siehe Abbildung \ref{configuration}) abgeleitet. 
Die Menge der Schichten wird mithilfe der oneTBB Funktion \texttt{parallel\_reduce} parallelisiert.

\begin{center}
    \lstinputlisting[language=C++,style=cpp,mathescape=true]{integral_2d_reduce.cpp}
    \captionof{figure}{Parallelisierung des 2D-Integrals}
    \label{2d_integral_parallel}
\end{center}

Dabei hat sich gezeigt, dass eine Blockgröße von 1 zu den besten Ergebnissen führt.
Die zu Berechnende Schicht kann nun in zwei Fälle unterteilt werden: Mit Singularität und ohne.

\subsubsection*{Fall 1: Keine Singularität in Schich $n$}

Dabei ist der Fall ohne Singularität der trivialere Fall:

\begin{center}
    \lstinputlisting[language=C++,style=cpp,mathescape=true]{integral_2d_trivial.cpp}
    \captionof{figure}{Schicht $n$ ohne Singularität}
    \label{2d_integral_trivial}
\end{center}


Die Funktion \textit{get\_partial\_integral} ist im Grunde genommen eine Kopie des eindimenisionalen Falles. Dieser wird allerdings aus Performanzgründen als lokale Funkion implementiert.
Die Kosten eines indirekten Aufrufs des eindimenisionalen Falls haben sich in den Benchmarks sehr negativ dargestellt.


\subsubsection*{Fall 2: Singularität in Schicht $n$}

Diesem Fall gehen die Berechnungen des ersten Falles vorraus. 
Allerdings muss in dieser Situation 

\begin{center}
    \lstinputlisting[language=C++,style=cpp,mathescape=true]{integral_2d_singularity.cpp}
    \captionof{figure}{Schicht $n$ mit Singularität}
    \label{2d_integral_sing}
\end{center}
