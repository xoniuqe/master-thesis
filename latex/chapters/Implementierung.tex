\chapter{Implementierung}

\section{Verwendete Technologien}


Mit Stern markierte gerne etwas weiter ausführen
\begin{itemize}
    \item Armadillo
    \item GNU Scientific Library *
    \item Pybind (Eigen3) *
    \item OneTBB *
    \item Catch2
    \item Google Benchmarks
    \item CMake
    \item Visualstudio Profiler
    \item Gprof (malsehen ob das noch genutzt wird)
    \item Valgrind*
\end{itemize}



Nicht genutzte Alternativen

Gründe noch aufführen
\begin{itemize}
    \item SIMD intrinsics * (mal sehen ob das sinnvoll ist)
    \item Boost numeric
    \item OpenCL
    \item OpenMP
    \item CUDA
\end{itemize}

\subsection{GNU Scientific Library}

Gerade bei der Frage nach dem Framework für Linalg und numerischer Integration lässt sich nicht abschätzen was die beste Lösung ist. Es gibt
so viele Frameworks und Bibliotheken, dass es nicht möglich ist alle gegeneinander Abzwuwägen.

Die Entshceidung GSL und Armadillo basier auf: Aramdillo ist einfach einzubinden und unkompliziert in der Anwendung.
GSL hat eine gute Perfomance. Der Verglecih lief mit Boost wobei einfache Tests zeigten das GSL schneller ist und mit weitaus weniger Aufwand in das Projekt integriert werden kann.
(Boost integration ist furchtbar!)


\subsection{Armadillo}

Kurze Einführng (1 Seite max?)

\subsection{Intel Threading Building Blocks}

Darstellung der beschleunigten Stellen, bzw beispiele wie man damit was parallelsiert + kurze einführenng

\subsection{Valgrind}

Erklärung der genutzten Tools Cachegrind und Cachetool?

\subsection{Pybind}

Kurze erklärung wie es klappt und beispiel für einfaches Plugin


\section{Ausgewählte Codestellen}

\subsection{Gauss laguerre integration und Cauchy Integral Theoerem}

\lstinputlisting[language=C++,style=cpp]{gauss_laguerre.cpp}

\subsection{1D Integration}



\subsection{2D Integration}

Warum konnte 1d Integration hier nicht direkt genutzt werden?
=> Indirektion durch das 1d Integral war relativ hoch, das teste ich aber besser noch mal