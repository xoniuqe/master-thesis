\chapter{Entwurf}
In diesem Kapitel wird auf verschiedene Aspekte der Entwurfsphase eingegangen.
Zum einen wird dargelegt mit welcher Herangehensweise der Algorithmus umgesetzt wird und zum anderen werden die Entscheidungen über die verwendeten Technologien begründet.

\section{Anforderungen}

Ziel dieser Arbeit ist es das in \cite*[]{gasperini:hal-03209144} beschriebene Verfahren zu implementieren.
Dabei sind drei Anforderungsbereiche auszumachen, welche in diesem Abschnitt behandelt werden.


\subsection{Korrektheit}

Die trivialste Anforderung ist den beschriebenen Algorithmus korrekt zu implementieren.
Diese Anforderung ist erfüllt, wenn die Rechenresultate der Implementierung eine Genauigkeit der gleichen Größenordnung wie die MatLab-Implementierung von \cite*[]{gasperini:hal-03209144}.
Dort werden in Kapitel 6.1 (Tabelle 3) für die Parameter 
\begin{equation}
    A = \begin{pmatrix}
        0 & 0 \\
        2 & 0 \\
        0 & 2 \\
    \end{pmatrix}, b = \begin{pmatrix}
        0 \\ -1 \\ 0
    \end{pmatrix},
    r = \begin{pmatrix}
        0.6 \\ 0 \\0 
    \end{pmatrix},
    \theta = \begin{pmatrix}
        1 \\ 0 \\
    \end{pmatrix}
\end{equation}

die relativen Fehler des implementierten Verfahrens mit der Matlab $integral$ Funktion für verschiedene Wellenzahlen $k$ verglichen.
\begin{table}[ht]
    \centering
    \begin{tabular}{|l|l|}
    \hline
    k & Relativer Fehler \\ 
    \hline \hline
    100  & $1.44 × 10^{-15}$ \\
    500  &  $1.15 × 10^{-14}$ \\
    1000 &  $9.71 × 10^{-5}$ \\
    3000 &  $5.11 × 10^{-7}$ \\
    5000 & $1.31 × 10^{-8}$ \\  \hline
    \end{tabular}    
    \caption{Genauigkeit der Implementierung von \cite*[]{gasperini:hal-03209144}}
\end{table} 
Das Kriterium der Korrektheit wird als erfüllt angesehen wenn die Implementierung höchstens Fehler der gleichen Größenordnung liefert.

\pagebreak

\subsection{Bedienbarkeit}

Die Anforderung der Bedienbarkeit wird im Rahmen dieser Arbeit wie folgt definiert:

\begin{enumerate}
    \item Es wird eine C++ Bibliothek bereitgstellt welche Mithilfe des Build-Systesm CMake eingebunden werden kann. 
    \item Es wird eine oder mehrere Matlab-Mex Funktionen ausgeliefert welche, den Algorithmus in Matlab benutzbar machen. 
    \item Es wird ein Python-Modul bereitgesellt mit welchem die 1D und 2D Algorithmen genutzt werden können.
\end{enumerate}

Dabei ist es nicht das Ziel die beschriebenen Pakete offiziel auszuliefern, das eingehändige Compilieren der Bilbiothek und ggf. der Module ist erforderlich.

\subsection{Geschwindigkeit}\label{Performance}


Der implementierte Algorithmus darf nicht langsamer als die bereitgestellte Matlab-Implementierung sein.
Zu diesesm Zweck werden einige Auswertungen mit randomisierten Daten mithilfe von MatLab ausgeführt und direkt mit der vorhandenen MatLab-Implementierung verglichen.



\section{Architektur}

Die Anwendung wird in drei Module aufgeteilt. 
Die Hauptbibliothek \texttt{steepest\_descent}, in welcher der eigentliche Algorithmus implementiert wird, sowie eine API definiert sodass sich die 1D-Integration, die 2D-Integration und ein Funktion für das parallele Verarbeiten von mehreren 2D-Integrationen von aussen Ansprechen lassen.
Diese API-Endpunkte dienen den Python- und Matlab-Wrappern die nötigen Anknüpfpunkte. 

Das Python-Modul \texttt{stedepy}welches die in der Hauptbibliothek API-Endpunkte bereitstellt, sowie eine Möglichkeit die benötigten Gauss-Laguerre-Knoten vorab zu berechnen.
Das Matlab-Modul, welches sich aus mehreren Mex-Funktionen zusammen setzt welche die Funkionen der API bereitstellen.


Abgesehen von dieser Aufteilung wird keine Architektur vorab definiert und es wird ein iteratives Vorgehen durchgeführt.
Die einzelnen API-Aufrufe teilen nur wenige gemeinsame Daten:
\begin{itemize}
    \item Die Wellenzahl $k$,
    \item den Beobachtungspunkt $r$,
    \item die Anzahl an Knoten für das Gauss-Laguerre-Verfahren,
    \item sowie die gewünschte Auflösung im zweidimensionalen Fall
\end{itemize}
Diese werden, abgesehen von dem Beobachtungspunkt $r$, in eine Konfigurationsklasse zusammengefasst.


\subsection{Parameter der Integration}

Die Parameter einer Integration eines Dreiecks teilen sich wie folgt auf:

\begin{itemize}
    \item Das Dreieck als parametrisierties Einheitsdreieck, bestehend aus einer $2x3$-Matrix $A$ und einem Verschiebungsvektor $b$,
    \item der Beobachtungspunkt $r$,
    \item ein Richtungsvektor $\theta$
\end{itemize}

Zusätzlich wird noch der Beobachtungspunkt als Parameter mitgereicht, dies ermöglicht eine vollständige Kompabilität mit der Matlab-Implementierung.

\subsection{Functor-Objekte}\label{sec_functor}

Die Hauptkomponenten zur Integrationen werden als sogenannte \textit{Functor}-Objekte (Funktor) implementiert.
Diese Datenstruktur zeichnet sich dadurch aus, dass sie den C++-Funktionsaufrufsoperator \textit{\(\)} bereitstellen und wie einfache C++-Funktionen aufgerufen werden können.
Da Funktoren durch Klassen bzw. \textit{structs} definiert werden kann in ihnen ein Zustand gespeichert werden.
In Abbildung \ref{2d_integral_functor} ist die Header-Definition des Funktor für das berechnen des zweidimensionalen Falls gezeigt.
In den Konstruktoren werden Parameter übergeben welche für mehr als eine Integration invariant sind. In der Implementierung des Funktionsaufrufsoperators werden die im vorherigen Abschnitt definierten Parameter erwartet.

\begin{center}
    \lstinputlisting{2d_integral.hpp}
    \captionof{figure}{Funktordefinition der zweidimensionalen Integration}
    \label{2d_integral_functor}
\end{center}