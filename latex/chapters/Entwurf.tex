\chapter{Entwurf}
In diesem Kapitel wird auf verschiedene Aspekte der Entwurfsphase eingegangen.
Zum einen wird dargelegt mit welcher Herangehensweise der Algorithmus umgesetzt wird und zum anderen werden die Entscheidungen über die verwendeten Technologien begründet.

\section{Anforderungen}

Ziel dieser Arbeit ist es das in \cite*[]{gasperini:hal-03209144} beschriebene Verfahren zu implementieren.
Dabei sind drei Anforderungsbereiche auszumachen, welche in diesem Abschnitt behandelt werden.


\subsection{Korrektheit}

Die trivialste Anforderung ist den beschriebenen Algorithmus korrekt zu implementieren.
Diese Anforderung ist erfüllt, wenn die Rechenresultate der Implementierung eine Genauigkeit der gleichen Größenordnung wie die MatLab-Implementierung von \cite*[]{gasperini:hal-03209144}.
Dort werden in Kapitel 6.1 (Tabelle 3) für die Parameter 
\begin{equation}
    A = \begin{pmatrix}
        0 & 0 \\
        2 & 0 \\
        0 & 2 \\
    \end{pmatrix}, b = \begin{pmatrix}
        0 \\ -1 \\ 0
    \end{pmatrix},
    r = \begin{pmatrix}
        0.6 \\ 0 \\0 
    \end{pmatrix},
    \theta = \begin{pmatrix}
        1 \\ 0 \\
    \end{pmatrix}
\end{equation}

die relativen Fehler des implementierten Verfahrens mit der Matlab $integral$ Funktion für verschiedene Wellenzahlen $k$ verglichen.
\begin{table}[ht]
    \centering
    \begin{tabular}{|l|l|}
    \hline
    k & Relativer Fehler \\ 
    \hline \hline
    100  & $1.44 × 10^{-15}$ \\
    500  &  $1.15 × 10^{-14}$ \\
    1000 &  $9.71 × 10^{-5}$ \\
    3000 &  $5.11 × 10^{-7}$ \\
    5000 & $1.31 × 10^{-8}$ \\  \hline
    \end{tabular}    
    \caption{Genauigkeit der Implementierung von \cite*[]{gasperini:hal-03209144}}
\end{table} 
Das Kriterium der Korrektheit wird als erfüllt angesehen wenn die Implementierung höchstens Fehler der gleichen Größenordnung liefert.

\pagebreak

\subsection{Bedienbarkeit}

Die Anforderung der Bedienbarkeit wird im Rahmen dieser Arbeit wie folgt definiert:

\begin{enumerate}
    \item Es wird eine C++ Bibliothek bereitgstellt welche Mithilfe des Build-Systesm CMake eingebunden werden kann. 
    \item Es wird eine oder mehrere Matlab-Mex Funktionen ausgeliefert welche, den Algorithmus in Matlab benutzbar machen. 
    \item Es wird ein Python-Modul bereitgesellt mit welchem die 1D und 2D Algorithmen genutzt werden können.
\end{enumerate}

Dabei ist es nicht das Ziel die beschriebenen Pakete offiziel auszuliefern, das eingehändige Compilieren der Bilbiothek und ggf. der Module ist erforderlich.

\subsection{Geschwindigkeit}\label{Performance}


Der implementierte Algorithmus darf nicht langsamer als die bereitgestellte Matlab-Implementierung sein.
Zu diesesm Zweck werden einige Auswertungen mit randomisierten Daten mithilfe von MatLab ausgeführt und direkt mit der vorhandenen MatLab-Implementierung verglichen.



\section{Architektur}



\begin{itemize}
    \item Iteratives Design 
    \item In Tests hat sich gezeigt, dass zu viel Design zu schlechtere Performance führt (z.B. nutzen von Parameterobjekten oder zu viel indirektion für besser lesbaren code)
    \item Functor Objekte erläutern
    \item Numerische Integraiton mit GSL muss noch gelöst werden
\end{itemize}

%\section{Multithreading betrachtung des Algorithmus}

%Was kann man wo Parallelisieren?

%\section{Thread pool bzw. object pool}

