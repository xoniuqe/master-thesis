\chapter{Vorgehensweise zur Optimierung}

In diesem Kapitel wird darauf eingegangen wie die in Kapitel \ref{Performance} definierten Ziele erreicht werden.


\section{Grober Plan}

Damit das Ziel der schnellen Performance erreicht werden wird die folgende Vorgehensweise genutzt.

Zu Beginn wird der Algorithmus möglichst simpel Implementiert mit einem Fokus auf Korrektheit. 
Das Korrekteverhalten wird durch Unittests sichergestellt. Die korrekten Eregbnisse einzelner Tests können mithilfe der vorliegenden Matlab Implementierung verifiziert werden.

Nachdem in diesem ersten Schritt ein korrekter Algorithmus vorliegt kann mit der Optimierung begonnen werden.
Dazu werden folgende Methoden verwendet, welche in den jeweiligen Unterkapiteln eingegangen wird.

\begin{itemize}
    \item Profiling
    \item Hotpath-Analyse
    \item Manuelle Tests
    \item Parallelisierung isolierter Bereiche
\end{itemize}

\section{Profiling}

Mithilfe der Profiling-Werkzeuge der Entwicklungsumgebung Visual Studio lässt sich das Laufzeitverhalten einer Anwendung analysieren.
In dieser Arbeit wird dabei das Messen der Geschwindigkeit als primäre Metrik genutzt. 


\section{Hotpath-Analyse}

Die Ergebnisse des Profilings ermöglichen es den sogenannten Hotpath (häufig auch Critical-Path genannt) eines Algorithmus zu finden.
Das Ziel dieser Betrachtung ist es die Funktion(en) zu finden welche am größten zu der gemessenen Laufzeit beitragen. 
Die so gefundenen Funktion sind die besten Kandidaten für Optimierungsmaßnahmen, da jede andere Funkion weniger zu der Laufzeit beiträgt.


\section{Benchmarks}

Automatische Benchmarks mit Googlebenchmark => genereller Überblick

Implementiertes Matlab Plugin kann im direkten Verlgeich mit der origniallen Matlab Implementierung ausgeführt werden


\section{Manuelle Tests}

Manuelle TEsts + PRofiling haben gezeig,t dass in den TEsts der Hotpath wie erwartet der Fall ohne Singularität ist.
D.h. die Optimierung lohnt sich am meisten auf dem Hotpath! (Ambehls law oder wie hießt das?)

\section{Nicht berücksichtigte Optimierungen}

Einige mögliche Optimierungen werden in dieser Arbeit nicht betrachtet

\begin{itemize}
    \item Verschiedene Compiler
    \item GPU beschleunigung: Numerische integrationsverfahren nicht auf GPU implementiert
\end{itemize}

%\section{Architekturmodelle}

